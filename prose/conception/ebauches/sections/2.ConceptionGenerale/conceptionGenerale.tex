% WARNING: automatically generated file that may be overwritten or removed at any time

\section{Conception générale}

\newcommand\macroSuffix{}
\input{../animUML/conceptionGenerale/conceptionGenerale-macros}


\subsection{Architecture candidate}

\begin{figure}[H]
	\centering
	\includegraphics[scale=.5,max width=\textwidth,max height=.9\textheight]{../animUML/conceptionGenerale/conceptionGenerale-context}
	\caption{Architecture candidate}
	\label{fig:archiCand}
\end{figure}
Le diagramme de la \autoref{fig:archiCand} représente l'architecture candidate du système.
\input{sections/2.ConceptionGenerale/descriptionArchiCand/description_archi_candi.tex}

\subsection{Diagrammes de séquence}

\begin{figure}[H]
	\centering
	\includegraphics[scale=.5,max width=\textwidth,max height=.9\textheight]{../animUML/conceptionGenerale/conceptionGenerale-sequence-CU1_INIT}
	\caption{Diagramme de séquence \emph{CU1 INIT}}
	\label{fig:inter-CU1_INIT}
\end{figure}
Le diagramme de la \autoref{fig:inter-CU1_INIT} représente le diagramme de séquence \emph{CU1 INIT}.
\input{sections/2.ConceptionGenerale/descriptionSeq/description_CU1_INIT.tex}

\begin{figure}[H]
	\centering
	\includegraphics[scale=.5,max width=\textwidth,max height=.9\textheight]{../animUML/conceptionGenerale/conceptionGenerale-sequence-CU2_MAJ_EXPLO}
	\caption{Diagramme de séquence \emph{CU2 MAJ EXPLO}}
	\label{fig:inter-CU2_MAJ_EXPLO}
\end{figure}
Le diagramme de la \autoref{fig:inter-CU2_MAJ_EXPLO} représente le diagramme de séquence \emph{CU2 MAJ EXPLO}.
\input{sections/2.ConceptionGenerale/descriptionSeq/description_CU2_MAJ_EXPLO.tex}

\begin{figure}[H]
	\centering
	\includegraphics[scale=.5,max width=\textwidth,max height=.9\textheight]{../animUML/conceptionGenerale/conceptionGenerale-sequence-CU3_LANCER_EXPLO}
	\caption{Diagramme de séquence \emph{CU3 LANCER EXPLO}}
	\label{fig:inter-CU3_LANCER_EXPLO}
\end{figure}
Le diagramme de la \autoref{fig:inter-CU3_LANCER_EXPLO} représente le diagramme de séquence \emph{CU3 LANCER EXPLO}.
\input{sections/2.ConceptionGenerale/descriptionSeq/description_CU3_LANCER_EXPLO.tex}

\begin{figure}[H]
	\centering
	\includegraphics[scale=.5,max width=\textwidth,max height=.9\textheight]{../animUML/conceptionGenerale/conceptionGenerale-sequence-CU4_EXPLORER}
	\caption{Diagramme de séquence \emph{CU4 EXPLORER}}
	\label{fig:inter-CU4_EXPLORER}
\end{figure}
Le diagramme de la \autoref{fig:inter-CU4_EXPLORER} représente le diagramme de séquence \emph{CU4 EXPLORER}.
\input{sections/2.ConceptionGenerale/descriptionSeq/description_CU4_EXPLORER.tex}

\begin{figure}[H]
	\centering
	\includegraphics[scale=.5,max width=\textwidth,max height=.9\textheight]{../animUML/conceptionGenerale/conceptionGenerale-sequence-CU5_ETEINDRE_MARCO}
	\caption{Diagramme de séquence \emph{CU5 ETEINDRE MARCO}}
	\label{fig:inter-CU5_ETEINDRE_MARCO}
\end{figure}
Le diagramme de la \autoref{fig:inter-CU5_ETEINDRE_MARCO} représente le diagramme de séquence \emph{CU5 ETEINDRE MARCO}.
\input{sections/2.ConceptionGenerale/descriptionSeq/description_CU5_ETEINDRE_MARCO.tex}

\begin{figure}[H]
	\centering
	\includegraphics[scale=.5,max width=\textwidth,max height=.9\textheight]{../animUML/conceptionGenerale/conceptionGenerale-sequence-CU6_DEPLACEMENT_MANUEL}
	\caption{Diagramme de séquence \emph{CU6 DEPLACEMENT MANUEL}}
	\label{fig:inter-CU6_DEPLACEMENT_MANUEL}
\end{figure}
Le diagramme de la \autoref{fig:inter-CU6_DEPLACEMENT_MANUEL} représente le diagramme de séquence \emph{CU6 DEPLACEMENT MANUEL}.
\input{sections/2.ConceptionGenerale/descriptionSeq/description_CU6_DEPLACEMENT_MANUEL.tex}


\subsection{Types de données}

\begin{figure}[H]
	\centering
	\includegraphics[scale=.5,max width=\textwidth,max height=.9\textheight]{../animUML/conceptionGenerale/conceptionGenerale-datatypes}
	\caption{Diagramme des types de données}
	\label{fig:datatypes}
\end{figure}
Le diagramme de la \autoref{fig:datatypes} représente les types de données utilisés.
\newline
L'énumération EScreen possède les littéraux suivants :
\enumEScreenLiteralDescriptions
L'énumération EConnectionState possède les littéraux suivants :
\enumEConnectionStateLiteralDescriptions
L'énumération EAlgo possède les littéraux suivants :
\enumEAlgoLiteralDescriptions
L'énumération EMove possède les littéraux suivants :
\enumEMoveLiteralDescriptions
L'énumération StatusEnum possède les littéraux suivants :
\enumStatusEnumLiteralDescriptions

\subsection{Classes}

\subsubsection{Vue générale}

\begin{figure}[H]
	\centering
	\includegraphics[scale=.5,max width=\textwidth,max height=.9\textheight]{../animUML/conceptionGenerale/conceptionGenerale-classes}
	\caption{Diagramme de classes}
	\label{fig:classes}
\end{figure}
Le diagramme de la \autoref{fig:classes} représente les classes du système.

\subsubsection{La classe User}

Le diagramme de la \autoref{fig:class-User} représente la classe User.
\begin{figure}[H]
	\centering
	\includegraphics[scale=.5,max width=\textwidth,max height=.9\textheight]{../animUML/conceptionGenerale/conceptionGenerale-class-User}
	\caption{Diagramme de la classe User}
	\label{fig:class-User}
\end{figure}
\input{sections/2.ConceptionGenerale/descriptionClass/description_User.tex}

\paragraph{Attributs}
\classUserProperties
\paragraph{Services offerts}
\classUserOperations
\subsubsection{La classe GUIPhone}

Le diagramme de la \autoref{fig:class-GUIPhone} représente la classe GUIPhone.
\begin{figure}[H]
	\centering
	\includegraphics[scale=.5,max width=\textwidth,max height=.9\textheight]{../animUML/conceptionGenerale/conceptionGenerale-class-GUIPhone}
	\caption{Diagramme de la classe GUIPhone}
	\label{fig:class-GUIPhone}
\end{figure}
\input{sections/2.ConceptionGenerale/descriptionClass/description_GUIPhone.tex}

\paragraph{Attributs}
\classGUIPhoneProperties
\paragraph{Services offerts}
\classGUIPhoneOperations
\paragraph{Description comportementale}
\begin{figure}[H]
	\centering
	\includegraphics[scale=.5,max width=\textwidth,max height=.9\textheight]{../animUML/conceptionGenerale/conceptionGenerale-GUIPhone-SM}
	\caption{Machine à états de \emph{GUIPhone}}
	\label{fig:sm-GUIPhone}
\end{figure}
Le diagramme de la \autoref{fig:sm-GUIPhone} représente la machine à états de \emph{GUIPhone}.
\input{sections/2.ConceptionGenerale/descriptionMAE/description_MAE_GUIPhone.tex}
\subsubsection{La classe NetworkPhone}

Le diagramme de la \autoref{fig:class-NetworkPhone} représente la classe NetworkPhone.
\begin{figure}[H]
	\centering
	\includegraphics[scale=.5,max width=\textwidth,max height=.9\textheight]{../animUML/conceptionGenerale/conceptionGenerale-class-NetworkPhone}
	\caption{Diagramme de la classe NetworkPhone}
	\label{fig:class-NetworkPhone}
\end{figure}
\input{sections/2.ConceptionGenerale/descriptionClass/description_NetworkPhone.tex}

\paragraph{Attributs}
\classNetworkPhoneProperties
\paragraph{Services offerts}
\classNetworkPhoneOperations
\subsubsection{La classe NetworkRobot}

Le diagramme de la \autoref{fig:class-NetworkRobot} représente la classe NetworkRobot.
\begin{figure}[H]
	\centering
	\includegraphics[scale=.5,max width=\textwidth,max height=.9\textheight]{../animUML/conceptionGenerale/conceptionGenerale-class-NetworkRobot}
	\caption{Diagramme de la classe NetworkRobot}
	\label{fig:class-NetworkRobot}
\end{figure}
\input{sections/2.ConceptionGenerale/descriptionClass/description_NetworkRobot.tex}

\paragraph{Attributs}
\classNetworkRobotProperties
\paragraph{Services offerts}
\classNetworkRobotOperations
\subsubsection{La classe Archivist}

Le diagramme de la \autoref{fig:class-Archivist} représente la classe Archivist.
\begin{figure}[H]
	\centering
	\includegraphics[scale=.5,max width=\textwidth,max height=.9\textheight]{../animUML/conceptionGenerale/conceptionGenerale-class-Archivist}
	\caption{Diagramme de la classe Archivist}
	\label{fig:class-Archivist}
\end{figure}
\input{sections/2.ConceptionGenerale/descriptionClass/description_Archivist.tex}

\paragraph{Attributs}
\classArchivistProperties
\paragraph{Services offerts}
\classArchivistOperations
\subsubsection{La classe UIRobot}

Le diagramme de la \autoref{fig:class-UIRobot} représente la classe UIRobot.
\begin{figure}[H]
	\centering
	\includegraphics[scale=.5,max width=\textwidth,max height=.9\textheight]{../animUML/conceptionGenerale/conceptionGenerale-class-UIRobot}
	\caption{Diagramme de la classe UIRobot}
	\label{fig:class-UIRobot}
\end{figure}
\input{sections/2.ConceptionGenerale/descriptionClass/description_UIRobot.tex}

\paragraph{Attributs}
\classUIRobotProperties
\paragraph{Services offerts}
\classUIRobotOperations
\subsubsection{La classe AlgoManager}

Le diagramme de la \autoref{fig:class-AlgoManager} représente la classe AlgoManager.
\begin{figure}[H]
	\centering
	\includegraphics[scale=.5,max width=\textwidth,max height=.9\textheight]{../animUML/conceptionGenerale/conceptionGenerale-class-AlgoManager}
	\caption{Diagramme de la classe AlgoManager}
	\label{fig:class-AlgoManager}
\end{figure}
\input{sections/2.ConceptionGenerale/descriptionClass/description_AlgoManager.tex}

\paragraph{Attributs}
\classAlgoManagerProperties
\paragraph{Services offerts}
\classAlgoManagerOperations
\subsubsection{La classe Driver}

Le diagramme de la \autoref{fig:class-Driver} représente la classe Driver.
\begin{figure}[H]
	\centering
	\includegraphics[scale=.5,max width=\textwidth,max height=.9\textheight]{../animUML/conceptionGenerale/conceptionGenerale-class-Driver}
	\caption{Diagramme de la classe Driver}
	\label{fig:class-Driver}
\end{figure}
\input{sections/2.ConceptionGenerale/descriptionClass/description_Driver.tex}

\paragraph{Attributs}
\classDriverProperties
\paragraph{Services offerts}
\classDriverOperations
\paragraph{Description comportementale}
\begin{figure}[H]
	\centering
	\includegraphics[scale=.5,max width=\textwidth,max height=.9\textheight]{../animUML/conceptionGenerale/conceptionGenerale-Driver-SM}
	\caption{Machine à états de \emph{Driver}}
	\label{fig:sm-Driver}
\end{figure}
Le diagramme de la \autoref{fig:sm-Driver} représente la machine à états de \emph{Driver}.
\input{sections/2.ConceptionGenerale/descriptionMAE/description_MAE_Driver.tex}
\subsubsection{La classe Explorer}

Le diagramme de la \autoref{fig:class-Explorer} représente la classe Explorer.
\begin{figure}[H]
	\centering
	\includegraphics[scale=.5,max width=\textwidth,max height=.9\textheight]{../animUML/conceptionGenerale/conceptionGenerale-class-Explorer}
	\caption{Diagramme de la classe Explorer}
	\label{fig:class-Explorer}
\end{figure}
\input{sections/2.ConceptionGenerale/descriptionClass/description_Explorer.tex}

\paragraph{Attributs}
\classExplorerProperties
\paragraph{Services offerts}
\classExplorerOperations
\subsubsection{La classe Radar}

Le diagramme de la \autoref{fig:class-Radar} représente la classe Radar.
\begin{figure}[H]
	\centering
	\includegraphics[scale=.5,max width=\textwidth,max height=.9\textheight]{../animUML/conceptionGenerale/conceptionGenerale-class-Radar}
	\caption{Diagramme de la classe Radar}
	\label{fig:class-Radar}
\end{figure}
\input{sections/2.ConceptionGenerale/descriptionClass/description_Radar.tex}

\paragraph{Attributs}
\classRadarProperties
\paragraph{Services offerts}
\classRadarOperations
\subsubsection{La classe Mapper}

Le diagramme de la \autoref{fig:class-Mapper} représente la classe Mapper.
\begin{figure}[H]
	\centering
	\includegraphics[scale=.5,max width=\textwidth,max height=.9\textheight]{../animUML/conceptionGenerale/conceptionGenerale-class-Mapper}
	\caption{Diagramme de la classe Mapper}
	\label{fig:class-Mapper}
\end{figure}
\input{sections/2.ConceptionGenerale/descriptionClass/description_Mapper.tex}

\paragraph{Attributs}
\classMapperProperties
\paragraph{Services offerts}
\classMapperOperations
\subsubsection{La classe Locator}

Le diagramme de la \autoref{fig:class-Locator} représente la classe Locator.
\begin{figure}[H]
	\centering
	\includegraphics[scale=.5,max width=\textwidth,max height=.9\textheight]{../animUML/conceptionGenerale/conceptionGenerale-class-Locator}
	\caption{Diagramme de la classe Locator}
	\label{fig:class-Locator}
\end{figure}
\input{sections/2.ConceptionGenerale/descriptionClass/description_Locator.tex}

\paragraph{Attributs}
\classLocatorProperties
\paragraph{Services offerts}
\classLocatorOperations
\subsubsection{La classe Robot}

Le diagramme de la \autoref{fig:class-Robot} représente la classe Robot.
\begin{figure}[H]
	\centering
	\includegraphics[scale=.5,max width=\textwidth,max height=.9\textheight]{../animUML/conceptionGenerale/conceptionGenerale-class-Robot}
	\caption{Diagramme de la classe Robot}
	\label{fig:class-Robot}
\end{figure}
\input{sections/2.ConceptionGenerale/descriptionClass/description_Robot.tex}

\paragraph{Attributs}
\classRobotProperties
\paragraph{Services offerts}
\classRobotOperations

\subsection{Diagrammes d'activité}

