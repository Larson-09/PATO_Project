% WARNING: automatically generated file that may be overwritten or removed at any time

\section{Conception générale}

\newcommand\macroSuffix{}
\input{../animUML/conceptionGenerale/conceptionGenerale-macros}


\subsection{Architecture candidate}

\begin{figure}[H]
	\centering
	\includegraphics[scale=.5,max width=\textwidth,max height=.9\textheight]{../animUML/conceptionGenerale/conceptionGenerale-context}
	\caption{Architecture candidate}
	\label{fig:archiCand}
\end{figure}
Le diagramme de la \autoref{fig:archiCand} représente l'architecture candidate du système.
\input{sections/2.ConceptionGenerale/descriptionArchiCand/description_archi_candi.tex}

\subsection{Diagrammes de séquence}


\subsection{Types de données}

\begin{figure}[H]
	\centering
	\includegraphics[scale=.5,max width=\textwidth,max height=.9\textheight]{../animUML/conceptionGenerale/conceptionGenerale-datatypes}
	\caption{Diagramme des types de données}
	\label{fig:datatypes}
\end{figure}
Le diagramme de la \autoref{fig:datatypes} représente les types de données utilisés.
\newline
L'énumération screenID possède les littéraux suivants :
\enumscreenIDLiteralDescriptions

\subsection{Classes}

\subsubsection{Vue générale}

\begin{figure}[H]
	\centering
	\includegraphics[scale=.5,max width=\textwidth,max height=.9\textheight]{../animUML/conceptionGenerale/conceptionGenerale-classes}
	\caption{Diagramme de classes}
	\label{fig:classes}
\end{figure}
Le diagramme de la \autoref{fig:classes} représente les classes du système.

\subsubsection{La classe USER}

Le diagramme de la \autoref{fig:class-USER} représente la classe USER.
\begin{figure}[H]
	\centering
	\includegraphics[scale=.5,max width=\textwidth,max height=.9\textheight]{../animUML/conceptionGenerale/conceptionGenerale-class-USER}
	\caption{Diagramme de la classe USER}
	\label{fig:class-USER}
\end{figure}
\input{sections/2.ConceptionGenerale/descriptionClass/description_USER.tex}

\paragraph{Attributs}
\classUSERProperties
\paragraph{Services offerts}
\classUSEROperations
\subsubsection{La classe GUI_PHONE}

Le diagramme de la \autoref{fig:class-GUI_PHONE} représente la classe GUI_PHONE.
\begin{figure}[H]
	\centering
	\includegraphics[scale=.5,max width=\textwidth,max height=.9\textheight]{../animUML/conceptionGenerale/conceptionGenerale-class-GUI_PHONE}
	\caption{Diagramme de la classe GUI_PHONE}
	\label{fig:class-GUI_PHONE}
\end{figure}
\input{sections/2.ConceptionGenerale/descriptionClass/description_GUI_PHONE.tex}

\paragraph{Attributs}
\classGUI_PHONEProperties
\paragraph{Services offerts}
\classGUI_PHONEOperations
\subsubsection{La classe ARCHIVIST}

Le diagramme de la \autoref{fig:class-ARCHIVIST} représente la classe ARCHIVIST.
\begin{figure}[H]
	\centering
	\includegraphics[scale=.5,max width=\textwidth,max height=.9\textheight]{../animUML/conceptionGenerale/conceptionGenerale-class-ARCHIVIST}
	\caption{Diagramme de la classe ARCHIVIST}
	\label{fig:class-ARCHIVIST}
\end{figure}
\input{sections/2.ConceptionGenerale/descriptionClass/description_ARCHIVIST.tex}

\paragraph{Attributs}
\classARCHIVISTProperties
\paragraph{Services offerts}
\classARCHIVISTOperations
\subsubsection{La classe UI_ROBOT}

Le diagramme de la \autoref{fig:class-UI_ROBOT} représente la classe UI_ROBOT.
\begin{figure}[H]
	\centering
	\includegraphics[scale=.5,max width=\textwidth,max height=.9\textheight]{../animUML/conceptionGenerale/conceptionGenerale-class-UI_ROBOT}
	\caption{Diagramme de la classe UI_ROBOT}
	\label{fig:class-UI_ROBOT}
\end{figure}
\input{sections/2.ConceptionGenerale/descriptionClass/description_UI_ROBOT.tex}

\paragraph{Attributs}
\classUI_ROBOTProperties
\paragraph{Services offerts}
\classUI_ROBOTOperations
\subsubsection{La classe DRIVER}

Le diagramme de la \autoref{fig:class-DRIVER} représente la classe DRIVER.
\begin{figure}[H]
	\centering
	\includegraphics[scale=.5,max width=\textwidth,max height=.9\textheight]{../animUML/conceptionGenerale/conceptionGenerale-class-DRIVER}
	\caption{Diagramme de la classe DRIVER}
	\label{fig:class-DRIVER}
\end{figure}
\input{sections/2.ConceptionGenerale/descriptionClass/description_DRIVER.tex}

\paragraph{Attributs}
\classDRIVERProperties
\paragraph{Services offerts}
\classDRIVEROperations
\subsubsection{La classe RADAR}

Le diagramme de la \autoref{fig:class-RADAR} représente la classe RADAR.
\begin{figure}[H]
	\centering
	\includegraphics[scale=.5,max width=\textwidth,max height=.9\textheight]{../animUML/conceptionGenerale/conceptionGenerale-class-RADAR}
	\caption{Diagramme de la classe RADAR}
	\label{fig:class-RADAR}
\end{figure}
\input{sections/2.ConceptionGenerale/descriptionClass/description_RADAR.tex}

\paragraph{Attributs}
\classRADARProperties
\paragraph{Services offerts}
\classRADAROperations
\subsubsection{La classe MAPPER}

Le diagramme de la \autoref{fig:class-MAPPER} représente la classe MAPPER.
\begin{figure}[H]
	\centering
	\includegraphics[scale=.5,max width=\textwidth,max height=.9\textheight]{../animUML/conceptionGenerale/conceptionGenerale-class-MAPPER}
	\caption{Diagramme de la classe MAPPER}
	\label{fig:class-MAPPER}
\end{figure}
\input{sections/2.ConceptionGenerale/descriptionClass/description_MAPPER.tex}

\paragraph{Attributs}
\classMAPPERProperties
\paragraph{Services offerts}
\classMAPPEROperations
\subsubsection{La classe LOCATOR}

Le diagramme de la \autoref{fig:class-LOCATOR} représente la classe LOCATOR.
\begin{figure}[H]
	\centering
	\includegraphics[scale=.5,max width=\textwidth,max height=.9\textheight]{../animUML/conceptionGenerale/conceptionGenerale-class-LOCATOR}
	\caption{Diagramme de la classe LOCATOR}
	\label{fig:class-LOCATOR}
\end{figure}
\input{sections/2.ConceptionGenerale/descriptionClass/description_LOCATOR.tex}

\paragraph{Attributs}
\classLOCATORProperties
\paragraph{Services offerts}
\classLOCATOROperations
\subsubsection{La classe ROBOT}

Le diagramme de la \autoref{fig:class-ROBOT} représente la classe ROBOT.
\begin{figure}[H]
	\centering
	\includegraphics[scale=.5,max width=\textwidth,max height=.9\textheight]{../animUML/conceptionGenerale/conceptionGenerale-class-ROBOT}
	\caption{Diagramme de la classe ROBOT}
	\label{fig:class-ROBOT}
\end{figure}
\input{sections/2.ConceptionGenerale/descriptionClass/description_ROBOT.tex}

\paragraph{Attributs}
\classROBOTProperties
\paragraph{Services offerts}
\classROBOTOperations

\subsection{Diagrammes d'activité}

